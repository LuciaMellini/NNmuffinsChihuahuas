\section{Evaluation of the models}\label{3_evaluation}
In this section we evaluate the various neural networks according to their risk with respect to zero-one loss. We will estimate such measure using $5$-fold cross validation. We have chosen to round the predicted labels in $\left[0,1\right]$ to the nearest digit.

To have a broader view on the quality of the models we have carried out cross validation on the original dataset, formed by training, validation and test sets, and a dataset composed only by the training and test sets. In fact, seen that the validation set has been used to score the best hyperparameters for each of the models, one expects that the risk of the predictor would be higher seen that it has been calibrated on the validation set. 

We evaluate according to both the risk estimation versions the models obtained by setting the hyperparameters according to the best values extracted by training on the original dataset and the augmented one (see Section~\ref{hyperparameter_tuning}). All the results are gathered in Table~\ref{tab:riskEstimates}.

\begin{table}
    \begin{subtable}{\textwidth}
        \centering
        \begin{tabular}{cccc}        
            \textsl{Model 1} & \textsl{Model 2} & \textsl{Model 3} & \textsl{Model 4} \\
            \midrule
            0.185902 & 0.232737 & 0.187934 & 0.290220
        \end{tabular}
    \end{subtable}
    \bigbreak
    \begin{subtable}{\textwidth}
        \centering
        \begin{tabular}{cccc}        
            \textsl{Model\_aug 1} & \textsl{Model\_aug 2} & \textsl{Model\_aug 3} & \textsl{Model\_aug 4} \\
            \midrule            
            0.195198 & 0.174581 & 0.178637 & 0.183877
        \end{tabular}
        \subcaption{Original dataset}
        \label{tab:riskEst_originalDataset}
    \end{subtable}
    \bigbreak
    \bigbreak
    \begin{subtable}{\textwidth}
        \centering
        \begin{tabular}{cccc}        
            \textsl{Model 1} & \textsl{Model 2} & \textsl{Model 3} & \textsl{Model 4} \\
            \midrule            
            0.093892 & 0.458927 & 0.127285 & 0.458927
        \end{tabular}
    \end{subtable}\bigbreak
    \begin{subtable}{\textwidth}
        \centering
        \begin{tabular}{cccc}        
            \textsl{Model\_aug 1} & \textsl{Model\_aug 2} & \textsl{Model\_aug 3} & \textsl{Model\_aug 4} \\
            \midrule            
            0.097869 & 0.458927 & 0.129812 & 0.408024
        \end{tabular}
        \subcaption{Dataset without validation set}
        \label{tab:riskEst_datasetWithoutVal}
    \end{subtable}

    \caption{Risk estimates computed through $5$-fold cross validation}
    \label{tab:riskEstimates}
\end{table}

\subsection{Original dataset}
In this section we look at the risk estimations found by carrying out $5$-fold cross validation on the original dataset, composed by the training, test and validation sets. 

We note that generally the risk for the models for which the hyperparameters have been chosen based on the augmented dataset is lower than the ones obtained by training on the original dataset. This is expected, since the models have been exposed to different versions of the images in the dataset, so they are more equipped to classify new images.

\paragraph{Hyperparameters chosen based on original dataset}
In the case of the models with hyperparameters chosen by looking at the original training set the difference in quality between the various models is more evident. In particular we see that \textsl{Model 1} and \textsl{Model 3} perform slightly better. Looking back at the structure of the hypermodels (see ~\ref{hypermodels}), this indicates that the addition of a dense layer at the end of the network does not help in the classification task. We can explain this by observing that the knowledge about the images is extracted at the convolutional level, so an additional reworking of these features only adds noise to the prediction.
Also, comparing the risk estimates for \textsl{Model 2} and \textsl{Model 4} it is made evident that the additional convolutional layer followed by a max pooling layer does not better the quality of the prediction. This is possibly due to the fact that a less complex network better at grasping the characteristic of the images, as an instance of the Occam Razor.

\paragraph{Hyperparameters chosen based on augmented dataset}
Viceversa, seen the results for the models obtained by choosing the hyperparameters based on an augmented version of the dataset, there is no clear winner. This is possibly due to the fact that the models have been exposed to a more varied dataset, so the quality of the prediction is more balanced. We note a slight advantage of \textsl{Model\_aug 2} and \textsl{Model\_aug 3}, implying that the simplest and the more complex solutions are not ideal.


\subsection{Dataset without validation set}
In this section we look at the risk estimations found by carrying out $5$-fold cross validation on the images in the training and test set. 